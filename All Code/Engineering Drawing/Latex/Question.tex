\documentclass{article}
\usepackage{polyglossia}
\setdefaultlanguage{english}
\setotherlanguage{bengali}
\newfontfamily\bengalifont[Script=Bengali]{Kalpurush}

\title{আমার প্রথম লেটেক ডকুমেন্ট}
\author{আপনার নাম}
\date{\today}

\begin{document}
\maketitle

এটি হল আমার প্রথম বাংলা লেটেক ডকুমেন্ট।

\section{পরিচিতি}
এই ডকুমেন্টের একটি সহজ পরিচিতি।

\section{অনুক্রমণিকা}
আপনি সহজেই অনুক্রমণিকা তৈরি করতে পারেন:
\begin{itemize}
    \item আইটেম ১
    \item আইটেম ২
    \item আইটেম ৩
\end{itemize}

\section{গণিত}
লেটেক একটি বিশালভাবে গণিত টাইপসেটিং জন্য ব্যবহৃত হয়। আপনি ইনলাইনে অথবা একটি আলাদা সমীকরণ হিসেবে গণিতকে যোগ করতে পারেন:
\[ E=mc^2 \]

\end{document}
